\chapter*{Prologo}
Queste dispense del corso di Metodi Numerici ed Ottimizzazione sono state elaborate durante il corso dell'A.A. 2010/2011, tenuto dal prof. Giancarlo Bigi e dal prof. Roberto Bevilaqua, nel primo anno in cui il corso è stato attivato nel Corso di Laurea Magistrale in Informatica dell'Università di Pisa.

{\em Sono state preparate da alcuni studenti soprattutto per avere una base per studiare il sostanzioso programma,  quindi mancano di organicità e di un linguaggio uniforme. Inoltre nella versione attuale vi sono probabilmente molti errori}. La speranza è che i professori (con l'aiuto degli studenti dei prossimi anni) possano farlo diventare, se non un libro di testo, almeno una dispensa ben fatta.

Il lavoro principale è stato quello di prendere gli appunti delle lezioni in \LaTeX , fatto da Andrea Cimino. Gli appunti sono poi stati sistemati, rielaborati ed integrati da Andrea Cimino, Marco Cornolti, Emmanuel Marzini, Davide Mascitti, Lorenzo Muti, Marco Stronati.

Spesso abbiamo consultato il materiale on--line offerto dai prof. Bevilacqua (parte sui metodi numerici) e Bigi (parte sull'ottimizzazione).

Si ringraziano gli sviluppatori degli strumenti open--source che abbiamo utilizzato per scrivere questa dispensa, in particolare: \LaTeX, Texmaker, Inkscape, Subversion.

Il lavoro -- sia nella versione compilata che nei sorgenti \LaTeX -- è pubblicato sotto licenza Creative Commons Attribuzione--Non commerciale--Condividi allo stesso modo 3.0 Italia (CC BY-NC-SA 3.0).

Chiunque è libero di riprodurre, distribuire, comunicare al pubblico, esporre in pubblico, rappresentare, eseguire e recitare quest'opera.

Chiunque è libero di modificare quest'opera e ridistribuirla, sotto le seguenti condizioni:
\begin{description}
\item[Attribuzione] l'opera derivata deve citare gli autori originali.
\item[Non Commerciale] non si può usare quest'opera né un'opera derivata per fini commerciali.
\item[Condividi allo stesso modo] l'opera derivata deve essere rilasciata sotto una licenza uguale o compatibile con questa.
\end{description}

Maggiori informazioni su \url{http://creativecommons.org/licenses/by-nc-sa/3.0/it/deed.it}

\vspace{1cm}
\begin{flushright}
Pisa, 28 settembre 2011
\end{flushright}

\vfill
\byncsa A. Cimino, M. Cornolti, E. Marzini, D. Mascitti, L. Muti, M. Stronati

