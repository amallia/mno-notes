\chapter{Risoluzione di sistemi lineari}
$ Ax=b \qquad \begin{array}{l}
A \in \mathbb{C}^{n \times n}, \det A \neq 0 \\
x \in \mathbb{C}^n
b \in \mathbb{C}^n
\end{array} $








\section{Metodi diretti}
\subsection{\fdefn{Metodo di Gauss}}
\positive{\ftheo{costa $ O(\frac{2}{3}n^3) $}}

\negative{\ftheo{bassa stabilit\'a}}

\begin{proc}[Metodo di Gauss]
\begin{enumerate}
\item fattorizzo A con procedura LR (\ref{proc:fattorizzazioneLR})
\item pongo $ LRx = b $ e risolvo in sequenza i due sistemi $ Ly = b $ e $ Rx = y $ (semplici in quanto triangolari)
\end{enumerate}
\end{proc}


\subsubsection{\fdefn{Fattorizzazione LR}}
Pongo $ A = LR \qquad \begin{array}{l}
L \in \mathbb{C}^{n \times n} \text{, triangolare inferiore con } \forall i \in [1,n].l_{ii} = 1 \\
R \in \mathbb{C}^{n \times n} \text{, triangolare superiore}
\end{array} $ 

\askip

\ftheo{esiste unica $ \Leftarrow $} $ \forall i \in [1,n], A_i \text{ sottomatrice princ. di testa di ordine i }. \det A_i \neq 0 $

\askip

\begin{proc}[Fattorizzazione LR]\label{proc:fattorizzazioneLR}
\begin{enumerate}
	\item pongo $ A^{(n)} = E^{(n)} E^{(n-1)} \ldots E^{(0)}A $, dove
	\begin{itemize}
		\item $ E^{(i)} = I - \sigma_i u_i v_i^H $ tale che $ E^{(i)}A^{(i)}e_i = w \text{ tale che } \forall j \in (i,n]. w_j = 0 \text{ e } w_i = 1 $
		\item $ \Rightarrow \bm{\left[ \sigma_i u_i \right]} = \left\lbrace \begin{array}{ll}
				0 & \quad j \leq i \\
				\frac{x_j}{x_i} & \quad j > i
			\end{array} \right. \qquad \bm{v_i} = e_i $
	\end{itemize}
	\item Alla fine $ L = \prod_{i = 0}^{n-1} \left[ E^{(i)} \right]^{-1} \qquad R = A^{(n)} $
\end{enumerate}
\end{proc}




\subsection{\fdefn{Metodo di Householder}}
\negative{\ftheo{costa $ O(\frac{4}{3}n^3) $}}

\positive{\ftheo{alta stabilit\'a}, soprattutto con particolari strategie di pivotaggio}

\begin{proc}[Metodo di Householder]
\begin{enumerate}
\item fattorizzo A con procedura QR (\ref{proc:fattorizzazioneQR})
\item pongo $ QRx = b $ e risolvo in sequenza i due sistemi $ Qy = b $ e $ Rx = y $ (semplici in quanto unitari e triangolari)
\end{enumerate}
\end{proc}


\subsubsection{\fdefn{Fattorizzazione QR}}
Pongo $ A = QR \qquad \begin{array}{l}
Q \in \mathbb{C}^{n \times n} \text{, unitaria ed hermitiana} \\
R \in \mathbb{C}^{n \times n} \text{, triangolare superiore}
\end{array} $ 

\askip

\ftheo{esiste sempre}

\askip

\ftheo{unica a meno di moltiplicazione per matrici di fase (diagonali, elem. su diagonale di modulo 1)}

\askip

\begin{proc}[Fattorizzazione QR]\label{proc:fattorizzazioneQR}
\begin{enumerate}
	\item pongo $ A^{(n)} = _P^{(n)} P^{(n-1)} \ldots P^{(0)}A $, dove
	\begin{itemize}
		\item $ \bm{\alpha_i} = \frac{x_i}{|x_i|} $
		\item $ \bm{v_i} = A^{(i)}e_i - \alpha_i e_i $
		\item $ \bm{\beta_i} = \dfrac{2}{|| v_i ||_2^{\phantom{2}2}} $
		\item $ P^{(i)} = I - \beta_i v_i v_i^H $ tale che $ P^{(i)}A^{(i)}e_i = w \text{ tale che } \forall j \in (i,n]. w_j = 0 \text{ e } w_i = \alpha_i $
	\end{itemize}
	\item Alla fine $ Q = \prod_{i = 0}^{n-1} \left[ P^{(i)} \right]^{-1} \qquad R = A^{(n)} $
\end{enumerate}
\end{proc}




\subsection{\fdefn{Metodo di Cholesky}}
\positive{\ftheo{costa $ O(\frac{1}{3}n^3) $}}

\positive{\ftheo{alta stabilit\'a}, soprattutto con particolari strategie di pivotaggio}

\negative{si applica solo in casi molto particolari}

\begin{proc}[Metodo di Cholesky]
\begin{enumerate}
\item fattorizzo A con procedura $ LL^H $ (\ref{proc:fattorizzazioneLLH})
\item pongo $ LL^Hx = b $ e risolvo in sequenza i due sistemi $ Ly = b $ e $ L^Hx = y $ (semplici in quanto triangolari)
\end{enumerate}
\end{proc}


\subsubsection{\fdefn{Fattorizzazione $ LL^H $}}
Pongo $ A = LL^H \qquad L \in \mathbb{C}^{n \times n}, \forall i \in [1,n].l_{ii} > 0 $

\askip

\ftheo{esiste unica $ \Leftarrow $} A hermitiana definita positiva

\askip

\begin{proc}[Fattorizzazione $ LL^H $]\label{proc:fattorizzazioneLLH}
per ogni colonna $ j \in [1,n] $
\begin{itemize}
\item per $ i \in [1,j) $ pongo $ l_{ij} = 0 $
\item per $ i = j $ pongo $ l_{ij} = \sqrt{a_{ij} - \sum_{k=1}^{j-1}|l_{jk}|^2} $
\item per $ i \in (j,n] $ pongo $ l_{ij} = \dfrac{a_{ij} - \sum_{k=1}^{j-1}|l_{jk}|^2}{l_{jj}} $
\end{itemize}
\end{proc}








\section{Metodi iterativi}
\begin{proc}[Schema iterativo per sistemi lineari]\label{proc:schemaiterativosistemilineari}
\begin{enumerate}
\item Pongo $ A = M - N \qquad \begin{array}{l}
M \in \mathbb{C}^{n \times n}, det M \neq 0
N \in \mathbb{C}^{n \times n}
\end{array} $ in modo dipendente dal particolare algoritmo
\item Pongo $ P = M^{-1}N $ e $ q = M^{-1}b $
\item Scelgo $ x^{(0)} \in \mathbb{C}^n $
\item Itero $ x^{(i+1)} = Px^{(i)} + q $ fino a che la soluzione non mi soddisfa
\end{enumerate}
\end{proc}

\askip

\ftheo{converge $ \Leftrightarrow $} $ \rho(P) < 1 $

\askip

\ftheo{converge $ \Leftarrow $} $ \exists || \cdot ||. \; || P || < 1 $




\subsection{\fdefn{Metodo di Jacobi} e \fdefn{Metodo di Gauss-Seidel}}
\begin{proc}[Decomposizione di A]\label{proc:decomposizioneAjacgse}
Pongo
\begin{itemize}
\item $ D = \left[ \begin{array}{ll}
0 & \quad i < j \\
a_{ij} & \quad i = j \\
0 & \quad i > j
\end{array} \right]_{ij} $
\item $ B = \left[ \begin{array}{ll}
0 & \quad i < j \\
0 & \quad i = j \\
-a_{ij} & \quad i > j
\end{array} \right]_{ij} $
\item $ C = \left[ \begin{array}{ll}
-a_{ij} & \quad i < j \\
0 & \quad i = j \\
0 & \quad i > j
\end{array} \right]_{ij} $
\end{itemize}
\end{proc}

\askip

\begin{proc}[Metodo di Jacobi]
\begin{enumerate}
\item Pongo $ A = D - C - B $ seguendo la procedura \ref{proc:decomposizioneAjacgse}
\item Applico la prodecura \ref{proc:schemaiterativosistemilineari} con $ M = D $ e $ N = B + C $
\end{enumerate}
\end{proc}

\askip

\begin{itemize}
\item Schema iterativo: $ x^{(i+1)} = D^{-1}(B+C)x^{(i)} + D^{-1}b $
\item indichiamo la matrice P di iterazione con la lettera J
\end{itemize}

\askip

\begin{proc}[Metodo di Gauss-Seidel]
\begin{enumerate}
\item Pongo $ A = D - C - B $ seguendo la procedura \ref{proc:decomposizioneAjacgse}
\item Applico la prodecura \ref{proc:schemaiterativosistemilineari} con $ M = D - B $ e $ N = C $
\end{enumerate}
\end{proc}

\askip

\begin{itemize}
\item Schema iterativo: $ x^{(i+1)} = (D-B)^{-1}Cx^{(i)} + (D-B)^{-1}b $
\item indichiamo la matrice P di iterazione con la lettera G
\end{itemize}

\askip

\ftheo{Jacobi e Gauss-Seidel convergono $ \Leftarrow $} si verifica una delle seguenti
\begin{itemize}
\item A ha \textbf{predominanza diagonale forte} per righe o colonne
\item A ha \textbf{predominanza diagonale debole} per righe o colonne ed \'e \textbf{irriducibile}
\end{itemize}

\askip

A hermitiana, $ \forall i \in [1,n]. \; a_{ii} > 0 \quad \Rightarrow $ \\
\ftheo{Gauss-Seidel converge $ \Leftrightarrow $} A definita positiva

\askip

\ftheo{Teorema di Stein-Rosenberg} Sia A con $ A_{ii} \neq 0 $, $ J $ sua matrice di Jacobi con $ \forall i,j \in [1,n]. \; a_{ij} \geq 0 $, $ G $ sua matrice di Gauss-Seidel. Allora si verifica una delle seg.
\begin{itemize}
\item $ \rho(G) = \rho(J) = 0 $
\item $ \rho(G) \leq \rho(J) \leq 0 $
\item $ \rho(G) = \rho(J) = 1 $
\item $ \rho(G) \geq \rho(J) \geq 0 $
\end{itemize}

\askip

T matrice tridiagonale, $ \forall i \in [1,n]. \; t_{ii} \neq 0 $, con J matrice di Jacobi e G matrice di Gauss-Seidel \ftheo{$ \Rightarrow $} autovalori di G \emph{quadrati} degli autovalori di J